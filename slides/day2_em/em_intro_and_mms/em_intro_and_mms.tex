\pdfoutput=1
\documentclass[10pt]{beamer}

%STANDARD PREAMBLE
%https://tex.stackexchange.com/questions/68821/is-it-possible-to-create-a-latex-preamble-header
\usepackage{../../rsrc/beamer_preamble}

%% ALLOW FOR ITEMIZE ENVIRONMENTS WITH NO PRECEDING
% SPACING, IF DESIRED
% Reference: https://tex.stackexchange.com/questions/86054/how-to-remove-the-whitespace-before-itemize-enumerate
%\usepackage{enumitem}% http://ctan.org/pkg/enumitem 
\usepackage{paralist}

% RANDOM VARIABLE
\newcommand{\x}{X}
\newcommand{\y}{Y}

%SUMS
\newcommand{\sumn}{\sum_{i=1}^{n}}


% ASSIGNMENTS
\newcommand{\assign}{z} %hardassignments
\newcommand{\Assign}{Z}
\newcommand{\soft}{r} %soft assignments

%%%% SUPPORT MAKING \subsectionpage
\AtBeginSection{\frame{\sectionpage}}
\AtBeginSubsection{\frame{\subsectionpage}}


\title{Expectation Maximization}

\begin{document}

\maketitle
\begin{frame}{Table of contents}
  \setbeamertemplate{section in toc}[sections numbered]
  \setbeamertemplate{subsection in toc}[subsections numbered]
   \setbeamertemplate{subsection in toc}{\leavevmode\leftskip=3.2em\rlap{\hskip-2em\inserttocsectionnumber.\inserttocsubsectionnumber}\inserttocsubsection\par} %indents the subsections
  \tableofcontents
\end{frame}

\begin{frame}{Acknowledgements}
This slide deck borrows heavily from an excellent course on statistical ML by Peter Orbanz.  
\vfill
Another important resource was Christopher Bishop's Machine Learning textbook.
\end{frame}

\section{The k-means algorithm}

\begin{frame}{Clustering}
\begin{sblock}{Problem}
\begin{minipage}{0.6\textwidth}
\begin{itemize}
\item Given: Data $x_1, ..., x_n$.
\item Assumption: Each data point belongs to exactly one group or class. These groups are called \bf{clusters}.
\item Our task is to find the clusters, given only the data.
\end{itemize}
\end{minipage} 
\hfill
\begin{minipage}{0.3\textwidth}
\includegraphics[width=\textwidth]{images/clustering}
\end{minipage} 

\end{sblock}

\begin{sblock}{Representation}
Fror $K$ clusters, we encode assignments to clusters as a vector $\+\assign \in \set{1,...,K}^n$ where:
\[ \+\assign_i = k \iff x_i \text{ assigned to cluster } k \]
\end{sblock}

%\begin{sblock}{Clustering and classification}
%Clustering is the ``unsupervised" counterpart to classification.   There is no training data and no labels, only one, unlabeled data set. 
%\end{sblock}

\end{frame}



%\begin{frame}{Example: Image Segmentation}
%\scriptsize
%\begin{sblock}{Image Segmentation}
%\bf{Image segmentation} is the problem of partitioning an image into ``coherent" regions.  The problem is not well-posed: Its solution depends on the meaning of ``coherent".
%\end{sblock}
%
%\begin{sblock}{Example}
%\includegraphics[width=\textwidth]{images/image_segmentation}
%\end{sblock}
%
%\begin{sblock}{Segmentation as a clustering problem}
%\begin{itemize}
%\item For each pixel, place a small window around the pixel.  Extract features (measurements) from this window.  For the $i$-th pixel, represent measurements by a vector $x_i$.
%\item Compute a clustering of the data $x_1, ..., x_n$ with K clusters.
%\item Each cluster represents one segment.  In the images above, one cluster is colored blue, one green, one red.
%\end{itemize}
%
%\end{sblock}

%\end{frame}


\begin{frame}{A very simple clustering algorithm: K-means}
\scriptsize
\begin{sblock}{K-means algorithm}
\begin{itemize}

\item Randomly choose K ``cluster centers" (the ``means") $\mu_1^{(0)}, ..., \mu_K^{(0)} \in \R^d$
\item Iterate until convergence ($j$ = iteration number):
	\begin{enumerate}
	\scriptsize
	\item Assign each $x_i$ to the closest (in Euclidean distance) mean:
	\[ \assign_i^{(j+1)} := \arg\min_{k \in \set{1,...,K}} ||x_i - \mu_k^{(j)} || \]
	\item Recompute each $\mu_k^{(j)}$ as the mean of all points assigned to it.
	\[ \mu_k^{(j+1)} := \df{1}{|i : \assign_i^{(j+1)} = k |}  \ds\sum_{i : \assign_i^{(j+1)} = k} x_i\]
	\end{enumerate}
\end{itemize}

\end{sblock}

\begin{sblock}{Convergence Criterion}
For example: Terminate when the total change of the means satisfies:
\[  \ds\sum_{k=1}^K || \mu_k^{(j+1)} - \mu_k^{(j)} || < \tau \]
The threshold value $\tau$ is set by the user.
\end{sblock}

\end{frame}

\begin{frame}{Illustration}
\begin{center}
\includegraphics[width=.7\textwidth]{images/k_means}
\end{center}
\footnotesize Illustration of the  $K$-means algorithm using the re-scaled Old Faithful dataset (in green).  The initial choices for centers $\mu_1$ and $\mu_2$ are shown by the red and blue crosses, respectively.
\vfill
\hfill \tiny Image Credit: Christopher Bishop
\end{frame}


\begin{frame}{Application: Image Segmentation and Clustering}

\scriptsize 

\begin{sblock}{Image Segmentation}
 \bf{Image segmentation} is the problem of partitioning an image into ``coherent" regions.   The problem is not well-posed: Its solution depends on the meaning of ``coherent".
\end{sblock}

\begin{sblock}{$K$-means on images}
\begin{itemize}
\item Each pixel is treated as a separate point representing $\{R,G,B\}$ intensities.
\item Note: not sophisticated; spatial proximity ignored.
\end{itemize} 

\begin{center}
\begin{tikzpicture}
  % Put a picture at a tikz node. 
  %I'm not sure what the anchor does, exactly.
    \node[anchor=south west,inner sep=0] (image) at (0,0) {\includegraphics[width=.5\textwidth]{images/k_means_images}};
    % The scope environment induces relative coordinates to the picture.
    \begin{scope}[x={(image.south east)},y={(image.north west)}]
        \node[text=black, text width=2cm] at (.20, 1.2) {$K=2$};
        \node[text=black, text width=2cm] at (.45, 1.2) {$K=3$};
         \node[text=black, text width=2cm] at (.70, 1.2) {$K=10$};
        \node[text=black, text width=2cm] at (.95, 1.2) {original};
    \end{scope}
\end{tikzpicture}
\end{center}
\hfill \tiny Image Credit: Christopher Bishop \scriptsize

\end{sblock}
%\begin{itemize}
%\item Each pixel treated as a separate point representing $\{R,G,B\}$ intensities.
%\item \bf{Image segmentation} is the problem of partitioning an image into ``coherent" regions.    \tiny ($K$-means not sophisticated; spatial proximity ignored.) \scriptsize
%\item (Lossy) \bf{image compression} obtained by storing only the cluster identity $k$ and the \it{code-book vectors} $\+\mu_k$.
%\end{itemize}



\begin{sblock}{Utility for image compression}
\it{(Lossy) image compression} is a side effect; one can store only the cluster identity $k$ and the \it{code-book vectors} $\+\mu_k$.
\end{sblock}
%\begin{itemize}
%\item For each pixel, place a small window around the pixel.  Extract features (measurements) from this window.  For the $i$-th pixel, represent measurements by a vector $x_i$.
%\item Compute a clustering of the data $x_1, ..., x_n$ with K clusters.
%\item Each cluster represents one segment.  In the images above, one cluster is colored blue, one green, one red.
%\end{itemize}

\end{frame}

\begin{frame}{$K$-Means: Gaussian Interpretation}
	\scriptsize
\begin{sblock}{$K$ Gaussians}
Consider the following algorithm:
\begin{itemize}
\item Suppose each $\mu_k$ is the expected value of a Gaussian density $p(x|\mu_k, \mathbb{I})$ with unit covariance.
\item Start with $K$ randomly chosen means and iterate.
	\begin{enumerate}
	\scriptsize
	\item Assign each $x_i$ to the Gaussian under which it has the highest density.
	\item Given the assignments, fit $p(x|\mu_k, \mathbb{I})$ by maximum likelihood estimation of $\mu_k$ from all points assigned to cluster $k$.
	\end{enumerate}
\end{itemize}
\end{sblock}

\begin{sblock}{Comparison to $K$-means}
\begin{itemize}
	\scriptsize
\item Since the Gaussians are spherical with identity covariance, the density $p(x|\mu_k, \mathbb{I})$  is largest for the mean $\mu_j$ which is closest to $x_i$ in Euclidean distance.
\item The maximum likelihood estimator of $\mu_k$ is 
	\[ \mu_k^{(j+1)} := \df{1}{|i : \assign_i^{(j+1)} = k |}  \ds\sum_{i : \assign_i^{(j+1)} = k} x_i\]
	This is precisely the $K$-means algorithm!
\end{itemize}

\end{sblock}
\end{frame}

\begin{frame}{What next}
\begin{itemize}
\item We will discuss a more sophisticated version of $K$-means called the \it{Expectation-Maximization (EM) algorithm.}

\item EM gives
	\begin{enumerate}
	\item A better statistical explanation of what is going on. 
	\item A direct generalization to other distributions.  We can consider Gaussians with general covariance structure, or other distributions as well.
    \item Better support for the anomaly detection use case.
	\end{enumerate}
\end{itemize}

\end{frame}

\section{Mixture Models} 

\begin{frame}{Finite mixture models}


\begin{sblock}{Finite Mixture Model}
A finite mixture model is a distribution with density of the form
\[ \pi(x) = \ds\sum_{k=1}^K c_k \; p(x \cond \theta_k) \]
where $\sum_k c_k =1$ and $c_k \geq 0$.

\end{sblock}
\begin{sblock}{Example: Finite mixture of Gaussians}
\begin{center}
\includegraphics[width=.5\textwidth]{images/gmm}
\end{center}

\end{sblock}

\end{frame}

% GMM for anomaly detection

\begin{frame}{Illustration}
\begin{sblock}{Mixture of two Gaussians}
\begin{minipage}{0.4\textwidth}
\includegraphics[width=\textwidth]{images/gmm_weights_1}
\end{minipage} 
\hfill 
\begin{minipage}{0.5\textwidth}
The curve outlined in red is the mixture
\[ \pi(x) = 0.5 \, p(x \cond 0, 1) + 0.5 \, p(x \cond 2, 0.5) \]
where $p$ is the Gaussian density.  The vblue curves are the component densities.
\end{minipage} 

\end{sblock}
\vfill

\begin{sblock}{Influence of the weights}
\begin{minipage}{0.4\textwidth}
\includegraphics[width=\textwidth]{images/gmm_weights_2}
\end{minipage} 
\hfill 
\begin{minipage}{0.5\textwidth}
Here, the weights $c_1=c_2 =0.5$ above have been changes to $c_1 = 0.8$ and $c_2 = 0.2$.  The component distributions are the same as above. 
\end{minipage} 

\end{sblock}
\end{frame}

\begin{frame}{Sampling}
\scriptsize

\begin{sblock}{Sampling from a finite mixture}
For a finite mixture with fixed parameters $c_k$ and $\theta_k$, the two-step sampling procedure is:
\begin{enumerate}
\item Choose a mixture component at random. Each component $k$ is selected with probability $c_k$.
\item Sample $x_i$ from $p(x|\theta_k)$.
\end{enumerate}
\bf{Note}: We always repeat both steps, i.e. for $x_{i+1}$, we choose again choose a (possibly different) component at random.

\end{sblock}
\tiny \hfill Note that k-means does not support sampling new points.


\begin{center}
\includegraphics[width=\textwidth]{images/gmm_sampling}
\end{center} 


\end{frame}

\begin{frame}{Mixture estimation}
\footnotesize
\begin{sblock}{Maximum likelihood for finite mixtures}
Writing down the maximum likelihood problem is straightforward:
\[ (\widehat{\+c}, \widehat{\+\theta}) = (\widehat{c}_1, ..., \widehat{c}_K, \widehat{\theta}_1, ..., \widehat{\theta}_K) = \arg\max_{\+c, \+\theta} \ds\prod_{i=1}^n  \bp{ \ds\sum_{k=1}^K c_k \, p(x_1 | \theta_k)} \]

The maximality equation for the logarithmic likelihood is
\[ \df{\delta}{\delta (\+c, \+\theta) } \ds\sum_{i=1}^n \log \bp{ \ds\sum_{k=1}^K c_k \, p(x_1 | \theta_k)}  = 0 \] 
The component equation for each $\theta_k$ is:
\[ \ds\sum_{i=1}^n \df{c_k \frac{\delta}{\delta \theta_k} p(x_i|\theta_k)}{\sum_{k=1}^K c_k p(x_i|\theta_k)}  =0 \] 
Solving this problem is analytically infeasible (note that we cannot multiply out the
denominator, because of the sum over $i$). Even numerical solution is often difficult.
\end{sblock}

\end{frame}

\begin{frame}{Latent Variables}


\begin{sblock}{Cluster assignments}
\begin{itemize}
\item The mixture assumption implies that each $x_i$ was generated from one component.
\item For each $x_i$, we again use an \bf{assignment variable} $\assign_i \in \set{1,...,K}$ which encodes which cluster $x_i$ was sampled from.
\end{itemize}
\end{sblock}

% https://tex.stackexchange.com/questions/300098/draw-a-node-as-a-square-with-tikz
\begin{minipage}{.45\textwidth}
  \centering
  \scalebox{0.7}{
  \tikz[square/.style={regular polygon,regular polygon sides=4}]{ %
     \node[square, draw](pi){$\pi$}; %
     \node[latent, below = of pi](xi){$x_i$};
     \edge{pi}{xi}
      \plate[inner sep=0.25cm, xshift=0cm, yshift=0cm] {plate1} {(xi)} {$N$}; %
  }
 }
\end{minipage} \hfill
\begin{minipage}{.10\textwidth}
\[ \Rightarrow \]
\end{minipage} \hfill
\begin{minipage}{.45\textwidth}
  \centering
  \scalebox{0.7}{
  \tikz[square/.style={regular polygon,regular polygon sides=4}]{ %
     \node[square, draw](c){$\+c$}; %
      \node[square, draw, right = of c](mu){$\+\mu$}; %
     \node[latent, below = of c](zi){$\assign_i$};
     \node[latent, below = of mu](xi){$x_i$};
     \edge{c}{zi}
      \edge{mu}{xi}
     \edge{zi}{xi}
      \plate[inner sep=0.25cm, xshift=0cm, yshift=0cm] {plate1} {(xi)(zi)} {$N$}; %
  }
 }
\end{minipage} \hfill


\begin{sblock}{Latent Variables}
Since we do not know which component each $x_i$ was generated by, the values of the assignment variables are \it{unobserved}. Such variables whose values are not observed are called \bf{latent variables} or \bf{hidden variables}.
\end{sblock}
\end{frame}

\begin{frame}{Estimation with latent variables}

\begin{sblock}{Latent variables as auxiliary information}
If we knew the correct assignments $\assign_i$, we could:
\begin{itemize}
\item Estimate each component distribution $p(x|\theta_k)$ separately, using only the data assigned to cluster $k$.
\item Estimate the cluster proportions $c_k$ as $\widehat{c}_k = \frac{\text{\# points in cluster} k}{n}$
\end{itemize}

\end{sblock}
\begin{sblock}{EM algorithm: Idea}
The EM algorithm estimates values of the latent variables to simplify the estimation problem. EM altnernates between two steps:
\begin{enumerate}
\item Estimate assignments $\assign_i$  given current estimates of the parameters $c_k$ and $\theta_k$ ("E-step").
\item Estimate parameters $c_i$ and $\theta_k$ given current estimates of the assignments ("M-step").
\end{enumerate}

These two steps are iterated repeatedly.
\end{sblock}
\end{frame}


\begin{frame}{Representation of Assignments}
\footnotesize
We re-write the assignments as vectors of length $K$:

\[ \+x_i \, \text{in cluster } k \quad \text{as} \quad \Assign_i := 
\begin{bmatrix}
0  \\
\vdots  \\
0  \\
1 \\
0  \\
\vdots  \\
0 
\end{bmatrix} \leftarrow k\text{th entry}
 \]
 so $\Assign_{ik}=1$ if $x_i$ in cluster k, and $\Assign_{ik} =0$ otherwise. \\
 \vfill
 We collect the vectors into a matrix
 \[ 
 \+\Assign := 
\begin{bmatrix}
\Assign_{11} & ... &  \Assign_{1K}\\
\vdots & & \vdots \\
\Assign_{n1} & ... &  \Assign_{nK}\\
\end{bmatrix} \
 \]
Note: Rows = observations, columns = clusters \\
Row sums = 1, column sums = cluster sizes.
\end{frame}

\begin{frame}{E-Step}
\footnotesize
\begin{sblock}{Hard vs. soft assignments}
\begin{itemize}
\item The vectors $\Assign_i$ are ``hard assignments" with values in $\set{0,1}$ (as in $k$-means)
\item EM computes ``soft assignments" $\soft_{ik}$ with values in $[0,1]$.
\item Once the algorithm terminates, each point is assigned to a cluster by setting
\[ \assign_i = \arg\max_k \soft_{ik} \]
The vectors $\Assign_i$ are the the latent variables in the EM algorithm.  The $\soft_{ik}$ are their current estimates
\end{itemize}
\end{sblock}

\begin{sblock}{Assignment probabilities}
The soft assignments are computed as
\[ \soft_{ik} = \df{c_k \, p(x_i \cond \theta_k)}{ \sum_{l=1}^K c_l  \, p(x_i \cond \theta_l} \]
They can be interpreted as
\[  \soft_{ik} := \E [\Assign_{ik} \cond x_i, \+c, \+\theta] = \text{Pr} \set{x_i \text{ generated by component } k \cond \+c, \+\theta} \]
\end{sblock}
\end{frame}

\begin{frame}{M-Step (1)}

\begin{sblock}{Objective}
The M-step re-estimates $\+c$ and $\+\theta$.  In principle, we use maximum likelihood within each cluster, but we have to combine it with the use of weights $\soft_{ik}$ instead of $\Assign_{ik}$
\end{sblock}

\begin{sblock}{Cluster sizes}
If we know which points belong to which cluster, we can estimate the cluster proportions $c_k$ by counting points:
\[ \widehat{c}_k = \df{\text{\# points in cluster } k}{n} = \df{\sum_{i=1}^n \Assign_{ik}}{n}\]
Since we do not know $\Assign_{ik}$, we substitute our current best guess, which is the expectations $\soft_{ik}$
\[ \widehat{c}_k  = \df{\sum_{i=1}^n \soft_{ik}}{n}\]
\end{sblock}
\end{frame}


\begin{frame}{M-Step (2)}
\footnotesize
\begin{sblock}{Gaussian special case}
The estimation of the component parameters $\theta$ depends on which distribution we choose for $p$.  For now, we assume a Gaussian. 
\end{sblock}
\begin{sblock}{Component parameters}
We use maximum likelihood to estimate $\theta=(\mu, \Sigma)$.  We can write the MLE of $\mu_k$ as 
\[ \widehat{\mu}_k =  \df{1}{\text{\# points in cluster } k}  \ds\sum_{i : x_i \text{in} k} x_i = \df{\sum_{i=1}^n  \Assign_{ik} x_i} {\sum_{i=1}^n \Assign_{ik}}\]
By substituting current best guesses $(=\soft_{ik})$ again, we get
\[ \widehat{\mu}_k = \df{\sum_{i=1}^n  \soft_{ik} x_i}{\sum_{i=1}^n \soft_{ik}}\]
For the covariance matrices:
\[ \widehat{\Sigma}_k = \df{ \sum_{i=1}^n \soft_{ik} (x_i - \widehat{\mu}_k ) (x_i - \widehat{\mu}_k )^T }{ \sum_{i=1}^n \soft_{ik}} \]
\end{sblock}
\end{frame}

\begin{frame}{Notation Summary}
\begin{sblock}{Assignment probabilities}
\[ 
 \+\soft := 
\begin{bmatrix}
\soft_{11} & ... & \soft_{1K}\\
\vdots & & \vdots \\
\soft_{n1} & ... &  \soft_{nK}\\
\end{bmatrix} 
= \E \bb{
\begin{bmatrix}
\Assign_{11} & ... & \Assign{1K}\\
\vdots & & \vdots \\
\Assign_{n1} & ... &  \Assign_{nK}\\
\end{bmatrix} 
} =
\begin{bmatrix}
\E[\Assign_{11}] & ... & \E[\Assign{1K}]\\
\vdots & & \vdots \\
\E[\Assign_{n1}[ & ... &  \E[\Assign_{nK}]\\
\end{bmatrix} 
 \]
 
\end{sblock}
\begin{sblock}{Mixture parameters}
\[ \+\tau = (\+c, \+\theta), \quad \+c = \text{cluster proportions } \quad \+\theta = \text{component parameters} \]
\end{sblock}
\begin{sblock}{Iterations}
$\+\theta^{(j)}, \+\soft^{(j)}, ...$ = values in $j$th iteration 
\end{sblock}
\end{frame}

\begin{frame}{Summary: EM for Gaussian Mixture}
\footnotesize
\begin{sblock}{Gaussian special case}
\[  \theta = (\mu, \Sigma) \, \text(mean \& covariance) \quad p(x \cond \theta) = p(x \cond \mu, \Sigma) \, \text{(Gaussian density)}\]
\end{sblock}
\begin{sblock}{Algorithm}
The EM algorithm for a finite mixture of Gaussians looks like this
\begin{enumerate}
\item \bf{Initialize:} Choose (e.g., random) values $c_k^{(0)}$ and $\theta_k^{(0)}$.
\item \bf{E-Step:} Recompute the assignment weioght matrix as
\[ \soft_{ik}^{(j+1)} = \df{c_k^{(j)} p(x_i \cond \theta_k^{(j)}}{ \sum_{l=1}^K c_l^{(j)} p(x_i \cond \theta_l^{(j)}}\]
\item \bf{M-Step:} Recompute the proportions $c_k$ and parameters  $\theta=(\mu, \Sigma)$ as


\[ \mu_k^{(j+1)} = \df{\sumn \soft_{ik}^{(j+1) x_i}{ \sumn \soft_{ik}^{(j+1}}} \quad \text{and} \quad \Sigma_k^{(j+1)}= \df{ \sumn \soft_{ik}^{(j+1}   (x_i  -\mu_k^{(j+1})  (x_i  -\mu_k^{(j+1})^T}{ \sumn \soft_{ik}^{(j+1}}   \] 
\end{enumerate}
The E-Step and M-Step are repeated alternatingly until covergence criterion (e.g. threshold) is satisfied.
\end{sblock}
\end{frame}

\begin{frame}{EM: Illustration}

\begin{sblock}{EM for a mixture of two Gaussians}
\begin{center}
\includegraphics[width=.6\textwidth]{images/em_for_mixture_of_two_gaussians}
\end{center}

\end{sblock}

The algorithm fits both the mean and the covariance parameter.
\end{frame}

\begin{frame}{GMMs as Universal Approximators}

\begin{quote}
A Gaussian mixture model is a universal approximator of densities, in the sense that any smooth density can be approximated with any specific nonzero amount of error by a Gaussian mixture model with enough components.
\end{quote}

\hfill \tiny Ian Goodfellow et al. 2016
\vfill
\tiny More formally, if $\mathcal{P}(\R^d)$  is the set of probability Borel measures on $\R^d$ (with its Euclidean topology), then ``Gaussian mixtures" (a.k.a. convex combinations of Gaussian measures) are dense in $\mathcal{P}(\R^d)$ for the  weak*  topology.
\vfill \vfill \vfill
\tiny \bf{Q:} So why not use GMM's for everything?

\end{frame}
% A's: Known support,  computation efficiency / estimation power (e.g. in high dimensions)

\begin{frame}{Other Mixture Models}
The mixture components do not need to be Gaussian in order for the model to be well-posed, and for EM to estimate the parameters.   \\
\vfill 
\begin{sblock}{Example: mixture of two betas}
\begin{center}
\includegraphics[width=.6\textwidth]{images/mixture_of_betas}

\end{center}
\scriptsize Shown are two beta mixture models, each with two components.  The models have fixed components but differ in their mixture weights: $\+c = (.25, .75)$ vs. $\+c = (.75, .25)$.
\end{sblock}

\vfill \vfill \vfill
\tiny \textbf{Q:} Under what conditions might one use a mixture of betas rather than a mixture of Gaussians?
\end{frame}

\begin{frame}{EM for General Mixture Models}
\footnotesize
\begin{sblock}{Algorithm}
\begin{itemize}
\item \bf{E-step:} Recompute the assignment matrix $\soft_{ik}^{(j)}$ as
\[ \soft_{ik}^{(j+1)} = \df{c_k^{(j)}  p(x_i \cond \theta_k^{(j)}) }{ \sum_{l=1}^K c_l^{(j)}  p(x_i \cond \theta_l^{(j)})} \]
\item \bf{M-step:} Recompute $(c,\theta)$ as
\[ (c^{(j+1}, \theta^{(j+1)}) = \arg\max_{c, \theta} \bb{ \ds\sum_{ik} \soft_{ik}^{(j+1)} \log (c_k \, p(x_i \cond \theta_k)) } \]
\end{itemize}
\end{sblock}
\begin{sblock}{Convenient special case}
If the MLE of $p(x \cond \theta)$ is of the form $\widehat{\theta}_{\text{ML}} =\frac{1}{n} \sumn s(x_i)$ for some function $s$  the M-step computes the ``weighted maximum likelihood estimate":
\[ c_k^{(j+1)}  = \df{\sum_{i=1}^n \soft_{ik}^{(j+1)}}{n} \quad \text{and} \quad \theta_k^{(j+1)} = \df{\sumn \soft_{ik}^{(j+1)} s(x_i)}{ \sumn \soft_{ik}^{(j+1)}}\]
This is true for any distribution in the exponential family.
\end{sblock}
\end{frame}


\section{Expectation Maximization (more generally)}

\begin{frame}{Latent variable models}

\begin{itemize}
\item A \textit{parametric statistical model} may posit observed random variables ($\+x$), parameters ($\+\theta$), and latent random variables ($\+z$).   
\item The distinguishing feature between latent variables $\+z$ and parameters $\+\theta$ is that the dimensionality of $\+z$ increases with the size of the data set, whereas the dimensionality of $\+\theta$ does not.\footnote{A more technical definition can be provided via conditional independence.  For example, when there is one latent variable per observation, a latent variable satisfies $p(\+x_n,\+z_n \cond \+x_{-n}, \+z_{-n}, \+\theta) = p(\+x_n,\+z_n \cond \+\theta)$, where the $-n$ subscript refers to the set of variables besides the $n$th. In other words, the $n$th observation and $n$th latent variable is independent of all other observations and latent variables, given the model parameters.}  
\item 
 Note that some presentations refer to $\+z$ as local hidden variables and $\+\theta$ as global hidden variables.
\end{itemize}

\end{frame}

\begin{frame}{Frequentist estimation for latent variable models}
\footnotesize
\begin{itemize}
\item Latent variable models provide a \bf{complete data likelihood}  $p (\+x, \+z \cond \+\theta)$, where $\+z$ is unobserved. The model often factorizes as
\[   p (\+x,  \+z \cond \+\theta) = \ds\prod_{i=1}^n  p (x_i, z_i \cond \+\theta) \]
\item For frequentist latent variable models, the inferential goal is to compute $\+\theta_{\text{ML}}$, the maximum likelihood value of the parameter.  
\item Since $\+z$ is not observed, so one seeks to find 
\begin{equation}
\+\theta_{\text{ML}}:=  \text{argmax}_{\+\theta} \; p (\+x \cond \+\theta) = \text{argmax}_{\+\theta} \ds\int p(\+x,\+z \cond \+\theta) \wrt{\+z} 
\label{ml_latent_vars}
\end{equation}
\item In particular, one requires access to the \textit{marginal} likelihood
\begin{equation}
p(\+x \cond \+\theta) = \ds\int p(\+x,\+z \cond \+\theta) \wrt{\+z} 
\label{marginal_likelihood}
\end{equation}
\end{itemize}
\end{frame}






\begin{frame}{Expectation Maximization for Latent Variable Models}

The \bf{expectation maximization algorithm} estimates $\+\theta$ by attempting to maximize the marginal likelihood. 

The expectation maximization algorithm is 

\begin{equation}
\label{eqn:em_algorithm_appendix}
 \+\theta^{(t+1)} =  \text{argmax}_{\+\theta} \; \E_{p(\+z \cond \+x , \+\theta^{(t)})} \bigg[ \ln p(\+x, \+z \cond \+\theta) \bigg] 
 \end{equation}
\end{frame}

\begin{frame}{Convergence Properties}
\footnotesize
\begin{sblock}{Marginal likelihood}
\begin{itemize}
\item It can be shown that the marginal likelihood $p(\+x \cond \+\theta)$ always increases from each step to the next, unless $\+\theta$ is already a stationary point.
\item The theory guarantees only that the algorithm terminates at a stationary point.  That point can be a saddle point rather than a maximum (very rare)

\end{itemize}
\end{sblock}
\begin{sblock}{The real problem: Local maxima}
\begin{minipage}{.6\textwidth}
\begin{itemize}
\item EM is effectively a gradient method.
\item The maxima it finds are the \bf{local maxima of the log-likelihood}
\item There are no guarantees on the global quality of the solution: The global maximum may differ arbitrarily from the one we find.
\end{itemize}
\end{minipage}
\hfill
\begin{minipage}{.35\textwidth}
\begin{tikzpicture}
  % Put a picture at a tikz node. 
  %I'm not sure what the anchor does, exactly.
    \node[anchor=south west,inner sep=0] (image) at (0,0) {\includegraphics[width=.7\textwidth]{images/em_local_maxima}};
    % The scope environment induces relative coordinates to the picture.
    \begin{scope}[x={(image.south east)},y={(image.north west)}]
        \node[text=black, text width=1cm] at (0,1.2) {$\log p(\+x \cond \+\theta)$};
        \node[text=black, text width=1cm] at (1.2, 0.0) {$\+\theta$};
    \end{scope}
\end{tikzpicture}
\end{minipage}

\end{sblock}
\vfill \tiny \bf{Q:} So what can we do about this?
\end{frame}

\begin{frame}{EM in practice}
\footnotesize
\begin{sblock}{Comparing solutions}
\begin{itemize}
\item If $\+\theta$ and $\+\theta^\prime$ are two different EM solutions, we can always compute the log-likelihoods
\[ \ds\sum_i \log p(x_i \cond \+\theta) \quad \text{and} \quad \ds\sum_i \log p(x_i \cond \+\theta^\prime) \]
\item The solution with the higher likelihood is better.
\item This is a very convenient feature of EM: Different solutions are comparable. 
\end{itemize}
\end{sblock}
\begin{sblock}{Random restarts}
In practice, the best way to use EM is often
\begin{itemize}
\item Restart EM repeatedly with randomly (or intelligently) chosen initial values.
\item Compute the log-likelihoods of all solutions and compare them.
\item Choose the solution achieving maximal log-likelihood
\end{itemize}
\end{sblock}
\vfill
\tiny \bf{Q}: What would be an intelligent way to initialize?
\end{frame}


\begin{frame}{EM for Exponential Family Latent Variable Models}
\scriptsize 
%We see how this plays out in the exponential family.   
Let us assume that $(\+x, \+z)=((x_1,z_1), ..., (x_n, z_n))$ are $n$ independent samples from the same exponential family, where $\+x$ is observed data and $\+z$ is unobserved data.
Moreover, let us assume that the complete data likelihood is in the exponential family

\begin{align}
\label{eqn:exponential_family_complete_data_likelihood}
 p(\+x, \+z \cond \theta) = \ds\prod_{i=1}^n h(x_i, z_i) \exp \big\{ \eta(\theta)^T \sum_{i=1}^n t(x_i, z_i) - n \,a(\eta(\theta))\big\} 
 \end{align}

Here we want to find $\+\theta$ to optimize 
\[ f(\+\theta) =  \E_{p(\+z \cond \+x , \+\theta^{(t)})} \bigg[ \ln p(\+x, \+z \cond \+\theta) \bigg] \]

Following the logic of Section \ref{sec:ml_with_ef}, we determine that we should select $\theta^{(t+1)}$ such that
\[ \mu(\theta^{(t+1)}) = \df{1}{n} \ds\sum_{i=1}^n   \E_{p(\+z \cond \+x , \+\theta^{(t)})} t(x_i, z_i) \]
where $\mu := \E[t(x_1,z_1)]$ refers to the mean parametrization of the likelihood.

This is why an EM iteration is often described and  implemented as iteratively computing \bf{maximum likelihood with the expected sufficient statistics.}
\end{frame}

\section{Some Cybersecurity Applications of EM}
\subsection{Better malware ground truth}
\begin{frame}{Introduction}
\begin{itemize}
\item Commercial anti-virus software traditionally memorizes specific byte sequences (known as \textbf{signatures}) in the file contents of previously encountered malware. 
\item They could use these signatures to attempt to detect malware in the future.
\item Malware authors can evade signature-based detection in many ways; for instance, by
	\begin{itemize}
	\item tampering with existing malware signatures (sometimes flipping a single bit)
	\item  using obfuscation techniques (e.g. encryption or compression) to hide snippets of malicious code
	\item writing metamorphic malware
	\end{itemize}
\item As a result, classical AV detections have \bf{low false positive} rates \it{but also} \bf{low true positive} rates.
\end{itemize}
\end{frame}

\begin{frame}{Better Malware Ground Truth}
Kantchelian et al. (2015) examine the problem of aggregating the results of multiple anti-virus (AV) vendors’ detectors into a single authoritative ground-truth label for every binary
\vfill \vfill \vfill
\tiny Kantchelian, A., Tschantz, M. C., Afroz, S., Miller, B., Shankar, V., Bachwani, R., ... \& Tygar, J. D. (2015, October). Better malware ground truth: Techniques for weighting anti-virus vendor labels. In Proceedings of the 8th ACM Workshop on Artificial Intelligence and Security (pp. 45-56).
\end{frame}

\begin{frame}{Model}
\footnotesize
\begin{sblock}{Random variables}

\begin{minipage}{.6\textwidth}
\begin{align*}
X_{ij} & \in \set{0,1} && \text{ AV label (i = instance, j =vendor)} \\
Z_i & \in \set{0,1} && \text{ground truth label} \\
\alpha, \beta & && \text{vendor tp, fp rates} \\
\pi & && \text{malware prevalence} 
\end{align*}

In our terminology, $\+Z$ are latent variables and $\alpha, \beta$ are parameters.
\end{minipage}
\hfill
\begin{minipage}{.35\textwidth}
\begin{center}
\includegraphics[width=.7\textwidth]{images/kantchelian_pgm}
\end{center}
\end{minipage}


\end{sblock}

\begin{sblock}{Model}

\begin{minipage}{.65\textwidth}
\begin{align*}
X_{ij} \cond Z_{i}, \alpha_j, \beta_j &= 
\begin{cases}
\alpha_j&  \text{ if } Z_i  =1   \text{ and }  X_{ij} = 1  \;  \text{(TP)} \\
1 - \alpha_j&  \text{ if }  Z_i  =1    \text{ and } X_{ij} = 0     \; \text{(FN)} \\
\beta_j &  \text{ if }  Z_i  =0  \text{ and } X_{ij} = 1     \; \text{(FP)} \\
1 - \beta_j& \text{ if } Z_i  =0  \text{ and } X_{ij} = 0    \;  \text{(TN)} \\
\end{cases} \\
Z_i \cond \pi & \sim \text{Bernouli}(\pi) \\
\pi &\sim \text{Beta (symmetric)} \\
\alpha \cond \psi &\sim \text{Beta (assymmetric)} \\
\beta \cond \phi &\sim \text{Beta (assymmetric)} \\
\end{align*}
\end{minipage}
\hfill 
\begin{minipage}{.3\textwidth}
\begin{center}
\includegraphics[width=\textwidth]{images/assymmetric_beta_kantchelian}

\scriptsize Asymmetric (right skewed) priors were used for $\alpha, \beta$ since both are expected to be low based on prior domain knowledge. 
\end{center}

\end{minipage}

\end{sblock}

\vfill \vfill \vfill \bf{Rk:} This is actually a \textit{Bayesian} latent variable model, since the parameters are assumed to be random variables.  However, we do not focus on that point here.
\end{frame}

\begin{frame}{Estimation and Results}

\begin{itemize}
\item The model was fit using EM.
\item The model far outperforms a common baseline in estimating ground truth.
\begin{center}
\includegraphics[width=.6\textwidth]{images/kantchelian_performance}
\end{center}
\item Moreover, it accomplished this \it{despite being fully unsupervised}!  (No ground truth was available during training.)
\end{itemize}

\end{frame}

\subsection{Behavioral biometrics}
\begin{frame}{Keystrokes biometrics}

Gaussian mixture models (and other density models) have been applied to typing behavior to flag anomalous behavior.

\begin{center}
\includegraphics[width=\textwidth]{images/keystroke_dynamics}
\end{center}

\end{frame}

\end{document}




%THE ACTUAL KANTCHELIAN MODEL.
\begin{sblock}{Model}

\begin{minipage}{.65\textwidth}
\begin{align*}
X_{ij} \cond Z_{i}, \alpha_j, \beta_j &= 
\begin{cases}
\alpha_j&  \text{ if } Z_i  =1   \text{ and }  X_{ij} = 1  \;  \text{(TP)} \\
1 - \alpha_j&  \text{ if }  Z_i  =1    \text{ and } X_{ij} = 0     \; \text{(FN)} \\
\beta_j &  \text{ if }  Z_i  =0  \text{ and } X_{ij} = 1     \; \text{(FP)} \\
1 - \beta_j& \text{ if } Z_i  =0  \text{ and } X_{ij} = 0    \;  \text{(TN)} \\
\end{cases} \\
Z_i & \sim \text{Bernouli}(\pi) \\
\pi &\sim \text{Beta (symmetric)} \\
\alpha \cond \psi &\sim \text{Beta (assymmetric)} \\
\beta \cond \phi &\sim \text{Beta (assymmetric)} \\
\end{align*}
\end{minipage}
\hfill 
\begin{minipage}{.3\textwidth}
\begin{center}
\includegraphics[width=\textwidth]{images/assymmetric_beta_kantchelian}
Assymmetric Beta (right skewed) were used since 
\end{center}

\end{minipage}

\end{sblock}

\vfill \vfill \vfill \bf{Rk:} This is actually a \textit{Bayesian} latent variable model, since the parameters are assumed to be random variables.  However, we do not focus on that point here.
\end{frame}




\begin{frame}{Hyperparameters}
\begin{tikzpicture}
  % Put a picture at a tikz node. 
  %I'm not sure what the anchor does, exactly.
    \node[anchor=south west,inner sep=0] (image) at (0,0) {\includegraphics[width=\textwidth]{images/bd_ihmm_hyperparameters}};
    % The scope environment induces relative coordinates to the picture.
    \begin{scope}[x={(image.south east)},y={(image.north west)}]
        \node[text=blue, text width=2cm] at (.30, 1.2) {more states};
        \node[text=blue, text width=2cm] at (.50, 1.2) {more blocks};
         \node[text=blue, text width=2cm] at (.70, 1.2) {freer mvmt across blocks};
        \node[text=blue, text width=2cm] at (.90, 1.2) {$\downarrow$ variability in transition probabilities};
    \end{scope}
\end{tikzpicture}
\scriptsize Truncated Markov transition matrices (right stochastic) sampled from the BD-iHMM prior with various fixed hyperparameter values.  Highlighted hyperparameters yield the chief observable difference from the leftmost matrix.
\end{frame}

